\chapter{Kalibrierung}
  Kapitel ist eingeteilt in radiometrische und geometrische kalibrierung.\\
  Bild: DSLR + Objekt + Display-Kamera System + (beispielhafter) Marker in Szene\\
 
 \section{Radiometrisches Modell} 
 Bild: wie oben, jedoch ohne webcam  und marker; Lichtfluss (2 Wege) von einzelnen Display-Pixeln zum Sensor (einmal direkt und einmal ueber das Objekt)\\
 Modell vorstellen und erlaeutern\\
 Blockdiagramm: Foto <- response curve <- Kamera-Sensor <- crosstalk <- display radiance <- response curves <- 8 bit Framebuffer \\
  Diagramm erklaeren; Insb. Erklaeren dass es zum Uebersprechen der drei Farben kommt.\\ 
 Jetzt schrittweise die Kalibrierung der einzelnen Teile vorstellen (v.l.n.r)\\
 
 \subsection{Ansprechverhalten der DSLR} 
  Robertson et al (pfstools) verwendet; andere moeglichkeiten: debvec\\
  Graph: reponse curve der Canon 5D; Aufteilen in linear + rest\\
  Linearen Bereich und ungenauigkeit der >90\% erwaehnen.

 \subsection{Crosstalk} 
  NOTE: besserer Titel\\
  Gleichung: Kamera-Radiance = Matrix * display-Radiance \\
  Dies kann man invertieren; Kalibrationsvorgang erklaeren (drei Bilder, matrix berechnen, invertieren);
 
 \subsection{Ansprechverhalten des Displays}
  Bild: display erzeugt ausschnitt einer environment map; lichtkegel von display zum ursprung\\
  Eingehendes Licht im Ursprung muss gemessen werden; darum aufstellen der Kamera an richtiger position  \\
  Bild: Zeichnung des Display Kalibrationsaufbaus \\
  
  Beschreibung der Kalibrationsvorgangs: 32x Greyscreen, per-pixel response curve linear interpoliert  \\
  Limitierung der lin. interpol.: Tendentiell immer zu niedrige Werte; kompliziertere Kurven sind moeglich falls ein kleinerer Fehler erwuenscht wird.  \\
  Anmerken dass aufgrund der verwendeten HDR Beleuchtungsmethode sich fehler in der response curve nicht so stark auf das Ergebniss auswirken.\\  
  Graph: Verlauf dreier Reponse curves eines Displays (zwei benachbarte patches + eins weiter entfernt um unterschiede bzw glatter verlauf zeigen zu koennen)\\
 NOTE: hier  zeigen dass alles funktioniert (zwischenergebnisse)? Z.B. drei Fotos vom Display: einmal bild direkt anzeigen, einmal gemappt ueber response curve und ohne crosstalk zu beachten, einmal response curve mit crosstalk beachten), differenz
 

\section{Geometrisches Modell}
 Bild: wie am Anfang vom Chapter, mit eingezeichneten Vektoren und Achsen der einzelnen koordinatensysteme  (World/Cam/Display/Marker)\\
  Translation-vector zwischen webcam und Display; wenn normal/eye-vector nicht parallel: transformations-matrix\\
 
\subsection{Intrinsische Parameter der Kameras}
  Bestimmung der Matrix und distortion-coefficients mittels opencv camera calibration; \\
  Kurze erklaerung der opencv methode\\
  Bild: Foto von schachbrett einmal mit und einmal ohne undistort.

\subsection{Display-Kamera System}
 NOTE: titel?
 Erste annahme: sehachse parallel zur display normalen; \\
 verifiziert und bestaetigt durch : schachbrettmuster parallel zu display aufgestellt , keine perspektivische verzerrrung im kamerabild sichtbar; \\
  Erwaehnen moeglicher automatischer verfahren (Spiegel oder zweite Kamera);\\

\subsection{Markierungen}
  Transformationsmatrix aller aller Markierungen zum welt-ursprung muss bekannt sein\\
  Wird von Hand definiert, verweis auf kapitel positionsbestimmung\\

    

    
  
