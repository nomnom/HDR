%Thema
In dieser Arbeit wird untersucht, wie man mit einem handelsüblichen Flachbildschirm eine Szene fotografisch beleuchten kann.
Spiegelnde und transparente Objekte lassen sich in einem Fotostudio nur eingeschränkt künstlich beleuchten, da sich die gesamte Umgebung in ihnen spiegelt und bricht.
Sie werden deshalb üblicherweise in einem weißen Fotozelt mit diffusen Flächenlichtquellen beleuchtet.
Verwendet man stattdessen einen Flachbildschirm als Lichtquelle, so können diese Objekte auch unter einer detaillierten Umgebungsbeleuchtung fotografiert werden.
Das Kontrastverhältnis herkömmlicher Bildschirmtechnologien ist im Vergleich zu dem realer Beleuchtungsszenarien allerdings sehr klein. 
Zur Abhilfe kann man statt einem statischen Bild eine zeitvariante Sequenz einsetzen, womit sich die kleinste darstellbare Lichtmenge reduzieren, und der Dynamikbereich vergrößern lässt.
In dieser Arbeit wird ein System vorgestellt, das es ermöglicht, eine Szene mit einem Bildschirm nahezu vollständig, in hoher Auflösung und mit hohem Dynamikbereich zu beleuchten.
Als Lichtquelle kommt dabei ein Laptop mit einer eingebauten Webcam zum Einsatz, sodass eine Beleuchtung von Hand erzeugt werden kann.
Ein Benutzer bewegt ihn dazu auf einer Halbkugel rund um eine Szene und erzeugt stückweise einzelne Teile der einfallenden Beleuchtung.
Die Bildschirmposition wird dabei automatisch mit der Webcam anhand von geometrischen Markern berechnet, die rund um die Szene angeordnet sind.
Die einzelnen Teilbeleuchtungen werden von einer Digitalkamera fotografiert, und die Aufnahmen im Anschluss zu einer vollständig beleuchteten Szene aufaddiert.
Es wird dabei unter Anderem auch gezeigt wie sich der Dynamikbereich der Beleuchtung durch Einsatz einer zeitvarianten Sequenz erhöhen lässt, und wie sich solche Sequenzen konstruieren lassen.

