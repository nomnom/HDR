
\chapter{Verwandte Arbeiten} \label{verwandte_arbeiten}

 
 Das System das in dieser Arbeit vorgestellt wird ist eine Light-Stage (siehe Kapitel \ref{lightstages}).
 Im Gegensatz zu den üblichen Light-Stage-Konstruktionen werden die Lichtquellen jedoch nicht mit diskreten Lampen realisiert, sondern mit einem mobilen Bildschirm.
 Dadurch kann zwar zu jedem Zeitpunkt immer nur ein Teil der einfallenden Beleuchtung erzeugt werden, dafür aber mit einer extrem hohen Auflösung.
 Angenommen, die Fläche einer Light-Stage-Lampe beträgt $7cm^2$ und der Bildschirm hat eine Pixeldichte von 96 DPI.
 Verwendet man einen Bildschirm, so lassen sich auf der selben Fläche $10^5$ Pixel unterbringen.
 Durch die sehr hohe Auflösung ergeben sich neue Anwendungsfelder, bei denen herkömmliche Light-Stages nicht eingesetzt werden können. 
 Die Zusammenhänge zu den unterschiedlichen Light-Stage-Konstruktionen werden in Abschnitt \ref{rel:lightstages} behandelt.
 
 Diese Arbeit hat auch Ähnlichkeiten mit den  \emph{Environment-Matting}-Verfahren.
 Sie ermöglichen es, ein reales Objekt derart von dem Hintergrund (engl: ``Backdrop'') zu separieren, dass es realistisch vor einem beliebigen anderen Hintergrund gesetzt werden kann. 
 Dazu wird der Lichtfluss zwischen Hintergrund, Objekt und Kamera modelliert und rekonstruiert, sodass beispielsweise auch die Lichtbrechung in transparenten Materialien erfasst werden kann, welche das Erscheinungsbild maßgeblich beeinflusst.
 Environment-Matting kann als Ergänzung zu Light-Stages gesehen werden: Mit diskreten Lampen kann nämlich kein Backdrop erzeugt werden. 
 Die Zusammenhänge zwischen dieser Arbeit und den Environment-Matting-Verfahren werden in  \ref{rel:envmatting} erklärt.

 
 
 \section{Light-Stages} \label{rel:lightstages}
  
% IBL 
  Fotografiert man eine Szene mit einer Kamera, so misst man im Endeffekt die Strahldichte die von vielen Punkten der Szene ausgeht.
  Wenn man die Szene dabei von unterschiedlichen Richtungen beleuchtet, so kann man das sogenannte Reflektanzfeld rekonstruieren.
  Aufgrund der additiven Eigenschaften des Lichttransports kann man die Aufnahmen linear kombinieren und die Szene unter einer neuen Beleuchtung ``rendern''.
  Dieses bildbasierte Renderingverfahren nennt sich ``Image-Based Relighting'', im Folgenden mit ``IBR'' abgekürzt.
  IBR hat die schöne intrinsische Eigenschaft, dass in den gerenderten Bildern so gut wie alle physikalischen Lichteffekte die in einem Material auftreten korrekt wiedergegeben werden können. 
  Bei der Aufnahme von Reflektanzfeldern verwendet man Light-Stages, die sich auf ganz unterschiedliche Arten konstruieren lassen.
 
%  Eine populäre Anwendung von IBR ist das Rendern von menschlicher Haut, welche durch ihren schichtweisen Aufbau und die detaillierte Oberflächenstruktur  nur schwer realistisch gerendert werden kann.
  Ein großer wissenschaftlicher Beitrag im Bereich IBL und Light-Stages wurde von Debevec und Hawkins geleistet.
  Sie waren die Ersten die das Reflektanzfeld eines menschlichen Gesichts vermessen, und unter einer neuen Beleuchtung gerendert haben \cite{Debevec_2000}. 
  Sie haben dazu eine Light-Stage entwickelt, die nur eine einzige Lichtquelle verwendet.
  Die Lampe sitzt auf einem beweglichen Ausleger der rund um die Szene rotiert werden kann und dessen Position zu jedem Zeitpunkt bekannt ist.
  Ganz ähnlich funktioniert die Free-Form Light-Stage von Masselus et al. \cite{Masselus_2002}. Sie besitzt ebenfalls nur eine Lichtquelle, die dabei jedoch von Hand gehalten und um die Szene herum bewegt wird. 
  Ihre Position kann dann  anhand der Lichtreflexion an einer Kugeloberfläche berechnet werden.
  Das gleiche Prinzip wird auch bei Fuchs et al. \cite{Fuchs_2005} angewendet. Sie erzeugen die einfallende Beleuchtung allerdings indirekt, über die diffuse Reflexion an den weißen Wänden eines Raumes.
  In der Arbeit von Mohan et al. \cite{Mohan_2005} wird das Licht stattdessen mit einem computergesteuerten Scheinwerfer erzeugt, der eine diffuse weiße Kiste von innen anstrahlt. 
  Anhand der Geometrie und der Scheinwerferausrichtung ist die Position der diffusen Lichtquelle zu jedem Zeitpunkt bekannt.

  Da für IBR die einfallende Beleuchtung zu jedem Zeitpunkt nur aus \emph{einer} Richtung kommen muss, kann sie auch mit einen mobiler Bildschirm erzeugt werden. 
  Das Verfahren, das in dieser Arbeit vorgestellt wird, kann hier also  verwendet werden.
 
% multiple lights -> nicht anwendbar 
  Es gibt jedoch auch Anwendungen in denen eine Szene aus vielen Richtungen \emph{gleichzeitig} beleuchtet werden muss, beispielsweise wenn die  Beleuchtung in Echtzeit erzeugt werden soll.
  Hier sind insbesondere die Arbeiten von Debevec et al. \cite{Debevec_2002} und Lamond et al. \cite{Lamond_2006} zu nennen. 
  Die Light-Stages, die sie dabei verwenden, bestehen aus zahlreichen computergesteuerten, farbigen Lichtquellen die (halb-) kugelförmig um die Szene angeordnet sind.
  Mit ihnen kann eine einfallende Beleuchtung in Echtzeit erzeugt werden. 
  Dadurch ist es möglich, eine Person unter einer zeitvarianten Beleuchtung zu filmen, weshalb das Prinzip in der Filmproduktion eingesetzt wird.
  Das Verfahren von Anrys et al. \cite{Anrys_2004} verwendet ebenfalls eine Light-Stage mit vielen Lichtquellen, allerdings um ein Objekt statisch, anhand einer Benutzervorgabe, zu beleuchten.
  Auch hier muss das Licht aus vielen Richtungen gleichzeitig in die Szene einfallen.
 
  Da mit einem mobilen Bildschirm immer nur ein Teil der einfallenden Beleuchtung erzeugt werden kann, lässt er sich für Echtzeitanwendungen nicht einsetzen.
   
 % more point sources   
  Durch die sehr hohe Auflösung eines Bildschirms ergeben sich allerdings auch neue Anwendungen, bei denen die üblichen Light-Stage-Konstruktionen nicht verwendet werden können.
  Ein spiegelndes Objekt lässt sich nämlich nicht mit diskreten Lampen beleuchten, da sie in der Oberfläche sichtbar werden.
  Verwendet man jedoch einen Bildschirm als Lichtquelle, so ist die Auflösung der Beleuchtung so hoch, dass die einzelnen Lichtquellen nicht mehr auf der Oberfläche auszumachen sind - es entsteht eine zusammenhängende Lichtfläche. 
  Mit einer hohen Auflösung kann man also eine Klasse von Objekten beleuchten, bei der diskrete Lampen versagen.

  

\section{Environment-Matting} \label{rel:envmatting}
  Environment-Matting kann man als Weiterentwicklung der normalen Matting-Verfahren ansehen.
  Bei dem einfachste Matting-Verfahren, dem Bluescreen-Matting \cite{Smith_1996}, wird der Hintergrund anhand des Farbtons vom Vordergrund getrennt.
  Das funktioniert allerdings nur, wenn sich die Szene farblich deutlich von der Hintergrundfarbe unterscheided. 
  Mit transparenten oder spiegelnden Objekte ist Matting deshalb nicht möglich.
 
  Das Environment-Matting erweitert das Matting-Prinzip, indem es den Lichtfluss zwischen Hintergrund, Szene und Kamera beachtet.
   Damit ist es beispielsweise möglich, ein Glasobjekt, das komplexe Lichtbrechungen aufweist, vor einen neuen Hintergrund zu platzieren.
  Auch Szenen mit feinen Strukturen wie Haare und Fell lassen sich so deutlich besser von einem Hintergrund trennen.
  Das Verfahren wurde erstmals von Zongker \cite{Zongker_1999} vorgestellt, der auch den Begriff ``Environment-Matting'' geprägt hat, und konnte im Laufe der Zeit verbessert werden \cite{Chuang_2000, Wexler_2002}.

  Environment-Matting stellen eine gute Ergänzung zu IBR dar: Mit einer Light-Stage lässt sich nämlich kein Hintergrund erzeugen.
  Eine Kombination beider Prinzipien wird in den Arbeiten von Matusik et al.  \cite{Matusik_2002} und Debevec et al. \cite{Debevec_2002} beschrieben.
   
  Environment-Matting kann durch das Verfahren, das im weiteren Verlauf beschrieben wird, zwar nicht ersetzt werden, das Endergebniss ist aber vergleichbar: 
  Durch die hohe Auflösung werden transparente und spiegelnde Objekte korrekt beleuchtet und ein Backdrop wird dabei automatisch erzeugt.
  Möchte man die Beleuchtung ändern, so muss man sie allerdings auch neu herstellen und aufnehmen, und kann sie nicht einfach neu berechnen.
